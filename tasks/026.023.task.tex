Спортивный самолёт массы $2000$ кг летит горизонтально
с ускорением $5$ м/с$^2$, имея в данный момент скорость $200$ м/с.
Сопротивление воздуха пропорционально квадрату скорости
и при скорости в $1$ м/с равно $0.5$ Н.
Считая силу сопротивления направленной в сторону, обратную скорости,
определить силу тяги винта, если она составляет угол $10^{\circ}$
с направлением полёта.
Определить также величину подъёмной силы в данный момент.
